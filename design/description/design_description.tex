\documentclass[a4paper,11pt]{article}
\usepackage[utf8]{inputenc}
\usepackage[T1]{fontenc}
\usepackage{lmodern}
\usepackage{geometry}
\geometry{margin=1in}
\usepackage{graphicx}
\usepackage{booktabs}
\usepackage{longtable}
\usepackage{hyperref}
\usepackage{amsmath}
\usepackage{listings}
\usepackage{parskip}

\title{Design Description -- Minor Smart Things}
\author{Tigo Goes}
\date{\today}

\begin{document}
\maketitle
\begin{abstract}
This document explains why each component in the project was selected and how it is used (electrically, mechanically and in firmware). The description covers the weather-station node and the actuator options used across the repository.
\end{abstract}

\tableofcontents
\newpage

\section{Scope and sources}
This description was prepared from the project's Bill of Materials and specification files. See the Bill of Materials: \texttt{design/bom/bom.tex} and the system specification: \texttt{design/spec.md}. The firmware mapping and manager responsibilities are taken from the header interfaces in the repository (SensorManager, DisplayManager, CommManager).

\section{System overview}
The project is a small distributed weather-station and actuator system. The weather-station measures temperature, humidity, wind speed, and ambient light. An actuator subsystem controls a sunshade; the system uses an MG90 hobby servo for shade actuation.

\section{Components --- rationale and usage}
The list below follows the Bill of Materials and the specification.

\subsection{ESP32 (Wi‑Fi / Bluetooth microcontroller)}
\textbf{Why chosen:} Combines Wi‑Fi, adequate I/O, and hardware PWM/LEDC channels; widely supported by PlatformIO/Arduino; low cost.

\textbf{How used:}
\begin{itemize}
\item Runs firmware responsible for sampling sensors, driving the display, running the shade logic, and publishing JSON to a server.
\item I2C master for the lux sensor and (optionally) the OLED.
\item Handles interrupts for the anemometer (Hall sensor) to measure wind speed.
\end{itemize}

\subsection{BH1750 (I2C lux sensor)}
\textbf{Why chosen:} Simple digital lux reading, factory calibrated, direct lux units, easy integration over I2C.

\textbf{How used:}
\begin{itemize}
\item Connected via I2C (SDA, SCL) to the ESP32 (see pin mapping).
\item Firmware reads lux periodically (1--5 s) and supplies values to the shade control logic.
\item Requires 4.7\,k pull‑ups to 3.3\,V on SDA/SCL if the breakout lacks them.
\end{itemize}

\subsection{DHT22 (Temperature \& Humidity)}
\textbf{Why chosen:} Inexpensive, easy-to-use digital sensor for T\&RH; adequate accuracy for the project goals.

\textbf{How used:}
\begin{itemize}
\item Single‑wire data line connected to an ESP32 GPIO with a 4.7--10\,k pull‑up.
\item Sampled on a slow cadence (e.g., once every 2--10\,s) to reduce noise and timing conflicts.
\item Managed by SensorManager (see \texttt{weatherStation/include/SensorManager.h}).
\end{itemize}

\subsection{A3144 (Hall‑effect sensor) --- anemometer}
\textbf{Why chosen:} Digital pulse output for robust counting; simple mounting; low cost.

\textbf{How used:}
\begin{itemize}
\item A magnet is mounted in the anemometer hub; each pass triggers the Hall switch.
\item Output wired to an interrupt‑capable ESP32 GPIO with a 10\,k pull‑up.
\item Debounce / edge‑counting handled in firmware; pulses per revolution configurable (spec default = 2).
\end{itemize}

\subsection*{Wind direction}
Wind direction sensing has been removed from this design. The system focuses on temperature, humidity, wind speed, and ambient light.

\subsection{OLED 0.96" SSD1306 (I2C)}
\textbf{Why chosen:} Compact, readable at low power, I2C interface simplifies wiring.

\textbf{How used:}
\begin{itemize}
\item Shows measured values and status messages; managed by DisplayManager.
\item Typically wired to the same I2C bus as BH1750; ensure address conflicts are resolved (0x3C/0x3D).
\end{itemize}

\subsection{Servo (MG90) --- sunshade actuator}
\textbf{Why chosen:} MG90 is a compact metal‑geared hobby servo selected because it provides the torque margin needed for the sunshade, reliable positional repeatability for many open/close cycles, and a small form factor that fits the designed mounts.

\textbf{How used:}
\begin{itemize}
\item Control via a PWM capable GPIO on the ESP32; use an LEDC channel for stable PWM.
\item Power the MG90 from a dedicated 5\,V supply capable of 1--2\,A peaks; common ground with ESP32.
\item Use a series 100\,$\Omega$ resistor on the signal line and a bulk cap (100--470\,µF) on the servo rail to smooth transients.
\end{itemize}

\subsection{Jumper cables, connectors and 3D‑printed parts}
\textbf{Why chosen:} Required for prototyping and housing. 3D printed brackets provide mounting and environmental shielding.

\textbf{How used:}
\begin{itemize}
\item Use quality connectors for external sensors; route servo cable through a cable gland for weatherproofing.
\item Print moving parts in PETG/ASA for outdoor durability (spec recommends this).
\end{itemize}

\subsection{Neodymium magnet}
\textbf{Why chosen:} Small strong magnet for triggering the Hall sensor on anemometer rotor.

\textbf{How used:}
\begin{itemize}
\item Secure magnet in rotor hub; choose size/grade to reliably trigger the Hall switch without unbalancing rotor.
\end{itemize}

\section{Electrical design notes and wiring}
\begin{itemize}
\item Logic domain: ESP32 operates at 3.3\,V. Tie sensor pull‑ups to 3.3\,V.
\item I2C: SDA $\rightarrow$ GPIO21, SCL $\rightarrow$ GPIO22 (default mapping in spec).
\item Use 0.1\,µF decoupling near each sensor and a 100\,µF bulk cap on the 5\,V rail.
\item Servo 5\,V supply must be able to handle peak current; do not power high current loads from the ESP32 3.3\,V regulator.
\item Common ground: All grounds (ESP32, 5\,V servo rail, motor drivers) must be common reference.
\end{itemize}

\section{Firmware mapping and responsibilities}
The repository provides manager classes that separate concerns:
\begin{itemize}
\item SensorManager: reads DHT22, BH1750, and wind speed edge counts; exposes read functions and periodic sampling (see \texttt{weatherStation/include/SensorManager.h}).
\item DisplayManager: updates the SSD1306 OLED with human‑readable information.
\item CommManager: packages sensor data into a JSON payload and publishes to server (see \texttt{weatherStation/include/CommManager.h}).
\end{itemize}

\subsection{ESP32 pin assignments (from specification)}
\begin{tabular}{llp{8cm}}
GPIO & Purpose & Notes \\
\midrule
21 & I2C SDA & BH1750 / OLED SDA \\
22 & I2C SCL & BH1750 / OLED SCL \\
16 & DHT22 DATA & 4.7--10\,k pull‑up \\
4 & Wind‑speed Hall input & attachInterrupt capable; 10\,k pull‑up \\
5 & Sunshade servo PWM & PWM signal, servo powered from 5\,V rail \\
2 & Status LED & 220\,$\Omega$ series resistor \\
\end{tabular}

\section{Calibration and testing}
\begin{itemize}
\item Wind speed: measure pulses per revolution and calibrate k\_speed against a reference (spec suggests placeholder k = 2.2\,m/rev).
\item Lux: compare BH1750 readings with a handheld lux meter and adjust logic thresholds (lux\_close\_threshold, lux\_open\_threshold).
\item Shade control: run repeated open/close cycles; ensure servos move to stored angles; tune hysteresis and dwell times to avoid oscillation.
\end{itemize}

\section{Alternatives and trade‑offs}
\begin{itemize}
\item Replace DHT22 with a higher‑accuracy sensor (e.g., SHT31) for better humidity/temperature performance.
\item Use TSL2561 instead of BH1750 for adjustable gains and better low‑light response.
\item Use a magnetic encoder or potentiometer for absolute shade position feedback if higher reliability is needed.
\end{itemize}

\section{Mechanical notes}
\begin{itemize}
\item 3D printed mounts should include drainage and mounting bosses for bearings.
\item Use stainless hardware for outdoor deployment and consider conformal coating for electronics.
\end{itemize}

\section{References}
Bill of Materials: design/bom/bom.tex \\
Specification: design/spec.md \\
Firmware interfaces: \texttt{weatherStation/include/SensorManager.h}, \texttt{weatherStation/include/DisplayManager.h}, \texttt{weatherStation/include/CommManager.h}

\end{document}