% Weather Station --- Comprehensive Manual Test Procedures
\documentclass[11pt,a4paper]{article}
\usepackage[utf8]{inputenc}
\usepackage[T1]{fontenc}
\usepackage[margin=1in]{geometry}
\usepackage{hyperref}
\usepackage{enumitem}
\usepackage{booktabs}
\usepackage{longtable}
\usepackage{tabularx}
\usepackage{amsmath}
\hypersetup{colorlinks=true,linkcolor=blue,urlcolor=blue}
\title{Weather Station --- Comprehensive Manual Test Procedures}
\author{Minor Smart Things --- Test Engineering \\ Author: Tigo Goes}
\date{\today}

\begin{document}
\maketitle

\section*{Purpose and scope}
This document contains detailed manual test procedures for bench and field verification of the Weather Station firmware and hardware. It is intended for engineers and technicians performing hands-on tests. Follow the preconditions and steps exactly; record artifacts for traceability.

\section{Test inventory}
The following hardware and firmware interfaces are covered:
\begin{itemize}[noitemsep]
\item DHT22 -- Temperature \& humidity sensor
\item BH1750 -- Ambient light / lux sensor
\item Hall-effect anemometer -- Wind-speed pulse output
\item Stepper motor -- Shade / vent control (PWM)
\item SSD1306 OLED display (I2C)
\item ESP-NOW -- peer-to-peer messaging
\item WiFi / HTTP client -- connectivity checks
\item Serial console and debug logging
\item Power supply and current draw measurements
\end{itemize}

\section{Tools and equipment}
Minimum recommended equipment:
\begin{itemize}[noitemsep]
\item Calibrated reference thermometer and hygrometer
\item Calibrated lux meter
\item Reference anemometer or controlled fan rig
\item Multimeter and current meter
\item Logic analyzer or frequency counter (for pulse capture)
\item Second ESP32 (peer) for ESP-NOW tests
\item USB-to-serial adapter and terminal program
\item Camera or phone for photos/screenshots
\end{itemize}

\section{Data capture and artifacts}
For each manual test record:
\begin{itemize}[noitemsep]
\item Date, time, tester name and board serial/ID
\item Firmware version or build identifier
\item Raw serial logs (plain text)
\item Measurement CSVs (timestamped)
\item Photos / screenshots of setup or display
\end{itemize}

\section{How to use acceptance criteria}
Acceptance criteria define PASS/FAIL per test. Use the following approach:
\begin{itemize}[noitemsep]
\item Define sample size (N) appropriate for the test. Typical N=10 for sensor stability checks; N=3--5 for quick verification.
\item Compute metrics: mean error (vs reference), standard deviation, maximum absolute error, failure rate (read errors).
\item Compare metrics against thresholds declared under each test. All thresholds must be met for PASS.
\item If a metric fails, record raw data and artifacts, then attempt a hardware swap (known-good module) to determine hardware vs firmware issue.
\item When reporting results, include the computed metrics and the raw CSV used to compute them.
\end{itemize}

\section{Manual test cases}

\subsection{DHT22 --- Temperature and humidity}
Purpose: Verify correct parsing and stability of DHT22 readings.

Preconditions:
\begin{itemize}[noitemsep]
\item DHT22 wired to the configured data pin and powered.
\item Serial console open and logging enabled.
\item Reference thermometer/hygrometer co-located within 5\,cm.
\end{itemize}

Steps:
\begin{enumerate}[noitemsep]
\item Start serial logging and note firmware identifier.
\item Trigger or wait for N=10 consecutive measurements at 2\,s intervals.
\item Save the readings to CSV with timestamps and raw sensor values.
\end{enumerate}

Measurements and thresholds:
\begin{itemize}[noitemsep]
\item Temperature: mean error within $\pm0.5\,^\circ$C.
\item Relative humidity: mean error within $\pm5\%$RH.
\item Read failure rate: $<$5\% in sample set.
\end{itemize}

Acceptance:
\begin{itemize}[noitemsep]
\item PASS if all thresholds are met and no parser exceptions are logged.
\end{itemize}

Artifacts: serial log, CSV, photo of sensor placement.

\subsection{BH1750 --- Ambient light (lux)}
Purpose: Verify I2C reads and conversion to lux.

Preconditions:
\begin{itemize}[noitemsep]
\item BH1750 connected to I2C and powered.
\item Reference lux meter calibrated and placed near sensor.
\end{itemize}

Steps:
\begin{enumerate}[noitemsep]
\item Capture N=5 readings at low, medium and high illuminance.
\item Compute lux = raw / 1.2 for each reading and record in CSV.
\end{enumerate}

Measurements and thresholds:
\begin{itemize}[noitemsep]
\item Mean error $<$10\% across tested range (10--10\,000 lux).
\item I2C transient errors tolerated up to 3 occurrences; persistent errors (5 consecutive) fail.
\end{itemize}

Acceptance: PASS if mean error $<$10\% and no persistent I2C errors.

Artifacts: CSV, serial log, photo of light source \& meter.

\subsection{Hall-effect anemometer --- Wind speed}
Purpose: Validate pulse counting, debounce and conversion to m/s.

Preconditions:
\begin{itemize}[noitemsep]
\item Pulse output connected to interrupt pin.
\item Known pulses-per-revolution (ppr) and rotor diameter available.
\end{itemize}

Steps:
\begin{enumerate}[noitemsep]
\item Run a controlled fan at three reference speeds; capture pps over fixed windows (e.g., 10\,s).
\item Compute speed: $v = \dfrac{\mathrm{pps}}{\mathrm{ppr}} \times \dfrac{\pi\times\mathrm{rotor\_diameter}}{\mathrm{measurement\_window\_s}}$.
\end{enumerate}

Measurements and thresholds:
\begin{itemize}[noitemsep]
\item Calibration error within $\pm10\%$ vs reference at each test point.
\item Debounce: ignore pulses shorter than 5\,ms.
\end{itemize}

Acceptance: PASS if all three test points within $\pm10\%$.

Artifacts: serial log, CSV, photos of rig.

\subsection{Stepper motor --- Position and response}
Purpose: Verify stepper motor control, accuracy, and response time.

Preconditions:
\begin{itemize}[noitemsep]
\item Stepper motor powered and mechanically coupled to load; position indicator (e.g., pointer and protractor, or rotary encoder) attached to measure angle.
\end{itemize}

Steps:
\begin{enumerate}[noitemsep]
\item Command the stepper to specific angles: 0°, 90°, and 180° (or full travel range as appropriate). Record the actual angle reached and the time taken to reach each position.
\end{enumerate}

Measurements and thresholds:
\begin{itemize}[noitemsep]
\item Position accuracy within $\pm5^\circ$ of target angle (or $\pm5\%$ of full travel, whichever is stricter).
\item Response time $\leq1$\,s to reach each commanded angle.
\end{itemize}

Acceptance: PASS if all positions meet accuracy and timing criteria.

Artifacts: video/photo, serial log of commands and positions.

\subsection{SSD1306 OLED --- Display verification}
Purpose: Verify initialization, rendering and stability.

Preconditions:
\begin{itemize}[noitemsep]
\item Display connected to I2C and powered.
\end{itemize}

Steps:
\begin{enumerate}[noitemsep]
\item Render a test pattern (text and shapes) and capture photo.
\item Update content at 5 updates/sec for 10 seconds and inspect for artifacts.
\end{enumerate}

Measurements and thresholds:
\begin{itemize}[noitemsep]
\item No persistent artifacts; text readable.
\item I2C errors: up to 2 transient allowed; persistent (5 consecutive) fail.
\end{itemize}

Acceptance: PASS if visual checks and error thresholds respected.

Artifacts: photos/video and serial logs.

\subsection{ESP-NOW --- Communications}
Purpose: Verify broadcast, peer messaging, acknowledgements, fragmentation and reassembly.

List of required items:
\begin{itemize}[noitemsep]
\item Device Under Test (DUT): the board running the firmware.
\item Peer device: second ESP32 configured to receive messages and log receipts.
\item Both devices with serial logging enabled.
\item Optional: WiFi access point to test coexistence and interference scenarios.
\end{itemize}

How to run ESP-NOW tests:
\begin{enumerate}[noitemsep]
\item Ensure peer MAC address is known and configured on DUT if testing peer-targeted messages.
\item Use the firmware's manual send command or console interface to trigger message bursts.
\item Capture serial logs on both DUT and peer with timestamps.
\end{enumerate}

Test A: Broadcast reliability
\begin{enumerate}[noitemsep]
\item Send 100 broadcast messages at a steady rate (e.g., 5 msg/s) with small payloads (32 bytes).
\item Peer counts and timestamps received messages.
\end{enumerate}

Metrics and thresholds:
\begin{itemize}[noitemsep]
\item Delivery rate $>$95\% under lab conditions.
\item Watch for duplicates and reorder; document occurrences.
\end{itemize}

Acceptance: PASS if delivery $>$95\% and no DUT crashes.

Test B: Peer sends with acknowledgements
\begin{enumerate}[noitemsep]
\item Send 100 messages to peer MAC with ack requested (if supported).
\item Verify ack receipts and measure round-trip latency for a sample subset.
\end{enumerate}

Metrics and thresholds:
\begin{itemize}[noitemsep]
\item Ack rate $>$95\%.
\item Typical latency within expected range (measure and document).
\end{itemize}

Acceptance: PASS if ack rate $>$95\% and no memory crashes.

Test C: Fragmentation \& large payload handling
\begin{enumerate}[noitemsep]
\item Send payloads larger than single-packet limit (split by firmware) and verify correct reassembly on peer.
\item Intentionally stress DUT queue to observe behavior when full.
\end{enumerate}

Metrics and thresholds:
\begin{itemize}[noitemsep]
\item Reassembly success rate and no data corruption.
\item DUT remains stable (no crashes) when queue fills; record dropped message policy observed.
\end{itemize}

Acceptance: PASS if reassembly works and DUT remains stable.

Artifacts: serial logs (DUT and peer), CSV of counts/timestamps, notes on RF environment.

\subsection{WiFi / HTTP --- Manual connectivity checks}
Purpose: Verify behavior during connect, disconnect, and queued delivery.

Preconditions:
\begin{itemize}[noitemsep]
\item WiFi credentials configured on DUT.
\item Local endpoint or test server available for manual verification.
\end{itemize}

Steps:
\begin{enumerate}[noitemsep]
\item With WiFi connected, trigger telemetry upload and observe server receipts.
\item Remove WiFi and trigger telemetry; observe DUT queueing behavior locally.
\item Reconnect WiFi and verify queued payloads are delivered; document times.
\end{enumerate}

Acceptance: PASS if queued payloads delivered after reconnection and DUT remains stable.

Artifacts: serial logs and server receipts.

\subsection{Power and stability tests}
Purpose: Measure current draw, brownout behavior and long-run stability.

Steps:
\begin{enumerate}[noitemsep]
\item Measure idle and active current draw with a current meter.
\item Run a soak test with periodic sensor readings and communications for 24 hours; capture periodic snapshots.
\item Induce brownout (gradual supply drop) and observe watchdog and recovery behavior.
\end{enumerate}

Measurements and thresholds:
\begin{itemize}[noitemsep]
\item Idle current within expected budget (document expected numbers per board).
\item No memory leaks or accumulative errors during soak; verify periodic heap statistics if available.
\end{itemize}

Acceptance: PASS if current draw within budget and no stability regressions over the soak period.

Artifacts: current traces, periodic logs, heap snapshots.

\subsection{I2C bus recovery and error handling}
Purpose: Verify the firmware recovers from transient I2C errors and bus lock states.

Steps:
\begin{enumerate}[noitemsep]
\item Cause a transient I2C error (pull SDA low briefly, or unplug/replug sensor) and observe behavior.
\item Verify firmware retries and recovers without crashing.
\end{enumerate}

Acceptance: PASS if firmware recovers and continues normal operation after transient I2C faults.

Artifacts: serial logs and notes on actions performed.

\subsection{Reset and watchdog behavior}
Purpose: Verify that unexpected faults trigger safe resets and that DUT recovers to a known good state.

Steps:
\begin{enumerate}[noitemsep]
\item Trigger a fault condition (e.g., deliberate memory allocation failure or watchdog kick test) and observe reset behavior.
\item After reset, verify firmware reinitializes sensors and resumes normal operation.
\end{enumerate}

Acceptance: PASS if reset occurs and DUT resumes expected operation without manual intervention.

Artifacts: pre/post reset logs and timestamps.

\section{Combined / acceptance scenario}
Purpose: End-to-end verification with all components active.

Steps:
\begin{enumerate}[noitemsep]
\item Boot DUT, wait until sensors stabilize (5 minutes).
\item Collect 3--5 readings per sensor and compare against references.
\item Send telemetry via ESP-NOW or WiFi and verify peers receive data.
\item Interact with the display; verify updates and responsiveness.
\end{enumerate}

Acceptance: PASS if all individual acceptance criteria are met and DUT remains stable during test.

Artifacts: consolidated logs, CSVs and photos.

\section{Test case recording template}
Use this template for each manual test:
\begin{longtable}{@{}p{3cm}p{10cm}@{}}
ID & TEST-xxx \\
Title & Short descriptive name \\
Purpose & What the test verifies \\
Preconditions & Hardware and configuration \\
Steps & Ordered steps taken \\
Measurements & Data collected (CSV filenames) \\
Acceptance criteria & Numeric/boolean criteria used \\
Result & PASS / FAIL \\
Artifacts & Filenames and locations \\
Notes & Observations, troubleshooting steps \\
\end{longtable}

\section{Recommended test order}
\begin{enumerate}[noitemsep]
\item Visual inspection and power checks
\item Sensor connectivity and basic reads (DHT22, BH1750, anemometer)
\item Display verification
\item Stepper motor functional test
\item Communications (ESP-NOW then WiFi)
\item Power and soak tests
\item Reset/Watchdog and I2C recovery tests
\end{enumerate}

\section{Troubleshooting guidance}
\begin{itemize}[noitemsep]
\item Swap suspected faulty hardware with a known-good module to isolate failures.
\item Reduce RF interference if ESP-NOW results are poor; move DUT and peer closer.
\item If I2C locks, power-cycle bus devices and observe recovery.
\end{itemize}

\section{References}
\begin{itemize}[noitemsep]
\item \href{./weatherStation/src/managers/SensorManager.cpp}{\texttt{weatherStation/src/managers/SensorManager.cpp:1}}
\item \href{./weatherStation/src/managers/CommManager.cpp}{\texttt{weatherStation/src/managers/CommManager.cpp:1}}
\item \href{./weatherStation/src/managers/EspNowManager.cpp}{\texttt{weatherStation/src/managers/EspNowManager.cpp:1}}
\end{itemize}

\section{Changelog}
\begin{itemize}[noitemsep]
\item 2025-10-02: Expanded manual test procedures with acceptance criteria and additional stability \& recovery tests.
\end{itemize}

\end{document}